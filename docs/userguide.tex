\documentclass[letterpaper]{article}
\usepackage[protrusion=true,expansion=true]{microtype}
\usepackage{fullpage}
\usepackage{pdfpages}
\usepackage{hyperref}
\usepackage{fontspec}
\usepackage{metalogo}

\usepackage[left=2cm,top=2cm,right=2cm,nohead,nofoot]{geometry} 

\setmainfont[Mapping=tex-text]{Minion Pro}
\setsansfont{Myriad Pro}
\setmonofont{Menlo}
\author{Robert I. Petersen}
\title{Robert I. Petersen CV}
\hypersetup{backref,
  pdftitle=JS Turtle User Guide,
  pdfauthor=Robert I. Petersen,
  pdfkeywords=javascript programing ,
  pdfsubject=js turtle,
  colorlinks=false} 
%\renewcommand*\thesubsection{\alph{subsection}}
\begin{document}
\section{Introduction}
Welcome to the user guide for JS Turtle.  The intention of
the JS Turtle program and this guide is to provide a platform for learning
about programming.  While the particular language used in the JS Turtle program
is JavaScript (thus the JS) the concepts presented such as looping, forks,
subroutines or functions are general concepts that are to be found in any
language.

This guide is written to be accesable to an older child or presented to young
children with the help of an adult.  In the latter case there is no requirement
that the supervising adult have any a priori programing knowledge, in fact the
uninitiated adult may find this guide an inspiration to learn programming as
well.

\section{Conventions}

\section{The Interface}

\section{Angles}
\end{document}
